\documentclass[12pt,a4paper]{report}
\usepackage{graphicx}
\usepackage{amsmath,amssymb}
\usepackage{geometry}
\usepackage{setspace}
\usepackage{fancyhdr}
\usepackage{hyperref}
\usepackage{listings} % For code listings
\usepackage{color}
\usepackage{url}

\geometry{margin=1in}
\setstretch{1.5}

% Custom commands
\newcommand{\HRule}{\rule{\linewidth}{0.5mm}}

% Title page
\title{
    \HRule \\[0.4cm]
    {\huge \bfseries  Efficient ZK Argument for Shuffle Implementation in Rust \\[0.4cm]}
    \HRule \\[1.5cm]
    \textsc{\Large Master's Thesis}\\[0.5cm]
    \textsc{\large University of Freiburg}\\[0.5cm]
}

\author{
    \Large Ahmet Ercem Bulut\\
    \Large Student ID: 5362638\\
    \Large Supervisor: Prof. Christian Schindelhauer
}

\date{August, 2024}

\begin{document}

% Title page
\maketitle
\thispagestyle{empty}
\newpage

% Abstract
\pagenumbering{roman}
\chapter*{Abstract}
\addcontentsline{toc}{chapter}{Abstract}
A shuffle operation in cryptography is an operation that takes committed and anonymous series of values and returns the original serie modified with a permuted order. Which is important in many real world scenarios(e-voting, mental card games).
Due to the plaintexts or data being encrypted, the correctness of a shuffle of commitments is not straight forward to verify. To overcome this problem an honest zero-knowledge verifier has been proposed to verify the correctness of a shuffle of homomorphic encryptions by Bayer and Groth(2012).
This argument for correctness combines two separate arguments(Multi-exponentiation Argument, Product Argument) to produce a Shuffle Argument for correctness. The implementation of these arguments are lacking in today’s literature of tools. With the use of Rust, which has enormous support from the cryptography community, and with it’s efficiency in runtime, we aim to provide an extensive and easy to use zero-knowledge proof
system.

\newpage

% Acknowledgements
\chapter*{Acknowledgements}
\addcontentsline{toc}{chapter}{Acknowledgements}
acknowledgements here.

\newpage

% Table of Contents
\tableofcontents

% List of Figures
\newpage
\listoffigures

% List of Tables
\newpage
\listoftables

% Main content
\newpage
\pagenumbering{arabic}

\chapter{Introduction}
\section{Background}
background here.

\section{Problem Statement}
problem statement here.

\section{Objectives}
research objectives here.

\section{Thesis Structure}
thesis structure here.

\chapter{Former Literature}
\section{Source Material}
\subsection{Efficient ZK Argument for Correctness of a Shuffle}
In 2012, Bayer and Groth\cite{bgshuffle} presented an algorithm with sublinear 
communication complexity for shuffling a deck of homomorphically encrypted values.

According to their findings, operations for an efficient sublinear size argument
show linearity in group elements when they are "in the exponent". \\
Using this adaptation, they constructed an efficient multi-exponentiation argument that a ciphertext
$C$ is the product of a set of known ciphertexts $C_1,...,C_N$ raised to a set of
hidden committed values. \\ 
By reducing this bottleneck sublinearly, the argument 
gains significant improvement in performance ($\mathcal{O}(\sqrt{N})$).\\
They also provide other optimization and minor improvements over the prover computations.

The algorithms used in the Bayer-Groth paper construct the backbone of our library,
with optimizations that are products of using native Rust and it's efficiency.
\section{Related Work}
Your related work here.

\section{Existing Solutions}
\subsection{Mental Poker}
Mental Poker\cite{mentalpoker} is a library aimed to implement a verifiable
mental poker game for research purposes. While it is also purely in Rust,
the library focuses more on the implementation and the efficiency of 
Barnett Smart Card Protocol\cite{Barnett}. They also differ in their choice
of cryptography primitives and go with arkworks curve points.\cite{arkworks}. 

\subsection{Bayer-Groth Mixnet}
A pure C++ implementation\cite{bgmixnet} of the protocol, for use in a messaging system.
Useful for us as well since they have in detail performance metrics we can
compare to. They use the same curve 25519 as ours and also give hardware specifications for the performances.\\
They have certain drawbacks such as: for some values of the parameter $m$ the 
verification fails, the row size $m$ should always be larger than the column size $n$ etc.. 
Our library works without these limitations as well.

\subsection{Practical Ad-Hoc Implementations}
While there are a few more implementations of the argument available online, 
they seem to either not be comprehensive enough for a comparison,
or small code used as injection for other projects. Which we decided not to
mention for brevity's sake.

\chapter{Methodology}
\section{Important Related Concepts}

\subsection{El Gamal Encryption}
\subsection{Pedersen Commitment}
\subsection{Homomorphic Property}

\section{Research Design}
\subsection{Ristretto Points}
A Ristretto Point is a cryptographic construction designed to create a
prime-order group from elliptic curves, 
enhancing the security and efficiency of cryptographic operations. 
In the context of mental card games, Ristretto Points 
enable players to prove knowledge or 
possession of certain cards without revealing the cards themselves.

\subsubsection{Prime-Order Group Properties}
Ristretto Points provide a prime-order group that simplifies 
mathematical operations and ensures predictable, 
secure behavior in cryptographic protocols.
A prime-order group eliminates issues related to cofactor multiplication, which can complicate the implementation of secure protocols.
In mental card games, this property ensures that each card, represented as a Ristretto Point, interacts securely and predictably within the proof system.

\subsubsection{Unique Encoding and Decoding}
One of the key features of Ristretto Points is their ability to encode and decode points on an elliptic curve in a way that eliminates ambiguities.
Each encoded point uniquely corresponds to a single group element, which is critical for maintaining the integrity of the cryptographic proofs.
This property ensures that each card in the mental card game, when encoded as a Ristretto Point, has a unique representation, preventing issues such as duplication or misidentification.

\subsubsection{Implementation of Choice}
We have chosen Dalek Cryptography's crypto tools framework curve25519-dalek\cite{dalek:curve}
as library of choice due to multiple reasons.
As the library supports the homomorphic properties of such group points, it also reduces the dimension of operations via discarding unnecessary operations such as scalar and point multiplication(which would be cofactor multiplication).\\
The library is also used prominently by the Rust Cryptography community in 
implementations of all kinds of cryptographic proofs and arguments.(bulletproofs, r1cs etc...). This gives us the advantage of being easily implementable into already existing Ristretto Point systems.


\chapter{Implementation}
\section{System Design}
system design here.

\section{Implementation Details}
implementation details here.

\chapter{Results}
results here.

\chapter{Discussion}
\section{Analysis of Results}
analysis of the results here.

\section{Implications}
discussion on implications here.

\chapter{Conclusion and Future Work}
\section{Conclusion}
conclusion here.

\section{Future Work}
suggestions for future work here.

% References
\newpage
\addcontentsline{toc}{chapter}{References}
\bibliographystyle{IEEEtran}
\bibliography{thesis}

% Appendix
\appendix
\chapter{Appendix A}
Your appendix content here.

\end{document}

